Figure \ref{fig:ui} presents parts of the user interface that investigators use to search the human trafficking knowledge graph and to visualize the results.
The user interface uses the keyword and faceted search interface paradigm, enabling investigators to search for terms that may appear in an escort advertisement, and to filter based on values extracted from the pages (Figure~\ref{fig:ldviews}.
Users may search for instances of the main classes in the ontology (WebPage, AdultService, Phone, etc.) and may filter based on any attribute connected via property paths to the selected class (e.g., filter Phone based on locality via the owner/availableAt/locality path).
The interface also offers visualizations to help investigators understand relationships and trends.
For example, investigators can view phone numbers on a map, they can see pie charts that summarize the number of locations and web sites associated with each phone, and they can view time-lines of offers using each phone number.

Supporting a rich user experience requires many queries on the knowledge graph and fast response times.
Below we summarize the types of queries that the knowledge graph must support, and show and example of each query type using a simplified SPARQL notation.