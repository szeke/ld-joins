Figure \ref{fig:ui} presents parts of the user interface that investigators use to query the human trafficking knowledge graph and to visualize the results.
The user interface uses the keyword and faceted search interface paradigm to enable investigators to search for any terms that may appear in an escort advertisement, and to filter based on attributes of the data (Figure~\ref{fig:ldviews}.
Users may search for instances of the main classes in the ontology (WebPage, AdultService, Phone, etc.) and may filter based on any attribute connected via property paths to the selected class (e.g., filter Phone using the localities via the owner/availableAt/locality path).
The interface also offers visualizations to help investigators understand relationships and trends.
They can view phone numbers on a map based on the locations of the associated offers, and can also see a summary of the number of locations and web sites associated with each phone.

The goal in the user interface is to give investigators as much insight as possible, which requires many queries on the knowledge graph and fast response times.
In the rest of this section we summarize the types of queries that we need to support, showing and example of each query type.