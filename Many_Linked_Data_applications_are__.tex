Many Linked Data applications are web portals that offer users query and visualization of RDF datasets constructed by integrating multiple data sources and mapping them to a domain model defined by one or several ontologies.
Notable examples include Open PHACTS \cite{Groth_Loizou_Gray_Goble_Harland_Pettifer_2014}, Bio2RDF \cite{callahan2013bio2rdf} and DIG \cite{szekely2015building}.
These applications use massive RDF datasets containing 100s of millions of RDF triples and require interactive query response

We propose a new approach to address the computational cost of RDF query processing to support web portal applications that use massive RDF datasets.
Our approach pre-computes the join operations needed in the queries to support the web portal, and indexes the resulting data so that no join operations are performed at query time.
We use the JSON-LD serialization of RDF \cite{Lanthaler:2012:UJC:2307819.2307827} to index the data in a NoSQL store, and enable clients to query the RDF using the native query language of the NoSQL store.

In section~\ref{sec:motivation} we present a taxonomy of the types of queries needed in Web portals and discuss how these queries can be used to implement web portals in a variety of domains.
We present details of our approach in section~\ref{sec:approach}, including a formal definition of the types of queries we propose to support, the approach for specifying and precomputing joins, and the approach to indexing the data in ElasticSearch, the NoSQL store we use in our implementation.
We evaluate our approach in section~\ref{sec:evaluation}, describing a new benchmark for evaluating query performance for web portal applications, and then compare the performance of our approach with centralized SPARQL and Linked Data Fragments.
We close with a discussion of related work in section~\ref(sec:related-work) and a discussion and directions for future work in section{sec:discussion}
