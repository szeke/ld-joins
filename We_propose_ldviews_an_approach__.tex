We propose \ldviews, an approach that enables data providers to define views on an RDF graph as directed acyclic graphs that can be materialized and indexed to support efficient queries.
Our approach uses the JSON-LD serialization of RDF \cite{Lanthaler:2012:UJC:2307819.2307827} to materialize the views as JSON documents, indexes the data in a NoSQL store, and enables clients to query the RDF using the native query language of the NoSQL store.
%
The approach provides significantly richer access to the graph than traditional parameterized SPARQL queries exposed as Web APIs and superior performance compared to equivalent SPARQL queries on the original graph.
Unlike Triple Pattern Fragments, \ldviews provides a sophisticated query language on the server, avoiding the need for clients to issue multiple requests to answer queries. 

We present a motivating example for our approach in section~\ref{sec:motivation}.
We present details of our approach in section~\ref{sec:approach}, including a formal definition of the types of queries we propose to support, the approach for specifying and pre-computing joins, and the approach to indexing the data in ElasticSearch, the NoSQL store we use in our implementation.
We evaluate our approach in section~\ref{sec:evaluation}, describing a new benchmark for evaluating query performance for web portal applications. 
Our evaluation shows that our approach vastly outperforms centralized SPARQL and Linked Data Fragments, enabling 1 second response times in a large web portal compared to XXX seconds in SPARQL and YYY seconds in Linked data Fragments.
We compare our work with other approaches in section~\ref{sec:related-work} and close with a discussion of the results and directions for future work in section~\ref{sec:discussion}.
