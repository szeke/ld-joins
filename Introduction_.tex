\section{Introduction}
There are several examples of applications that illustrate the benefits of Semantic Web technology to build data analysis applications that integrate data from heterogeneous sources.
Notable examples include Bio2RDF \cite{callahan2013bio2rdf} with 11 billion triples, Open PHACTS \cite{Groth_Loizou_Gray_Goble_Harland_Pettifer_2014} with 3 billion triples, Event Registry \cite{Leban_Fortuna_Brank_Grobelnik_2014} with 56 million documents and DIG \cite{szekely2015building} with 74 million documents.
These systems offer rapid response times, with queries and navigation responding within one 1 or 2 seconds, they have a community of users and provide high levels of availability.
Bio2RDF and Open PHACTS integrate RDF datasets, store their data in triple stores and use SPARQL to query the data.
Event Regsitry and DIG integrate extractions from Web documents, both adding extractions from over 100,000 new pages every day.
They both use semantic technologies to integrate the data, but neither uses a triple store or SPARQL due to scalability, performance and robustness considerations.

Despite substantial research on optimizing SPARQL query processing \cite{Pham2013}, the computation of query solutions remains a computationally intensive task for large datasets.
As a result, it challenging for application developers to achieve query response times of 1 or 2 seconds as users expect and as advocated in user interface guidelines \cite{nielsen1994usability}, especially when serving large numbers of users.
Application developers often spend a substantial amount of time optimizing queries to support interactive applications \cite{Loizou_Angles_Groth_2014}.
In addition, it is challenging for maintainers of SPARQL endpoints to guarantee the levels availability required in production applications such as the ones listed above.
For example, a recent survey \cite{buil2013sparql} revealed that less than half of the public endpoints reach an availability of 95\%.

Common approaches to address the performance and availability challenges involve restricting the types of queries that can be submitted to the SPARQL endpoint.
We can think of the different approaches as defining views over an RDF graph.
A simple approach is to define views of the graph as a parameterized SPARQL queries, similar to views in a relational database.
Open PHACTS uses this approach, exposing the parameterized queries to clients using a Web API.
The approach is very conservative as it limits clients to a small number of queries.
%
Triple Pattern Fragments \cite{Verborgh2014} provides simple views of an RDF graph, where each view is defined in terms of a triple pattern, where any of the subject, predicate or object roles can be a variable.
Queries consist of a single triple pattern, so servers can answer them with minimal computational resources.
The approach provides a client-side library that implements a subset of SPARQL in terms of Triple Pattern Fragments, thereby exposing a rich query language for clients, implemented on top of the simple, efficient views.
The approach guarantees high availability, but each query generates multiple requests to the server, resulting in significant data transfer between servers and clients and performance degraded below the advocated user interface guidelines . 

