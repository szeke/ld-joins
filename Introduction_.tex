\section{Introduction}
There are several examples of applications that illustrate the benefits of Semantic Web technology to build data analysis applications that integrate data from heterogeneous sources and that continuously evolve their schema to integrate new sources.
Notable examples include Open PHACTS \cite{Groth_Loizou_Gray_Goble_Harland_Pettifer_2014}, Bio2RDF \cite{callahan2013bio2rdf}, Event Registry \cite{Leban_Fortuna_Brank_Grobelnik_2014}and DIG \cite{szekely2015building}.
These applications use massive datasets: 
Open PHACTS \cite{Groth_Loizou_Gray_Goble_Harland_Pettifer_2014} has over 3 billion triples\footnote{http://semantics.cc/open-phacts-semantic-interoperability-drug-discovery};
our human trafficking dataset has over 5 billion triples that integrate data extracted from over 70 million web pages, adding over 100,000 new pages every day;
Event Registry has 56 million articles, also adding over 100,000 new articles every day.
These systems also offer rapid response times, with queries and navigation responding within one 1 or 2 seconds, they support a substantial number of users and high levels of availability.
All these systems use semantic technologies for integrating the data, but only Open PHACTS use a triple store.
The others use alternative technologies due to scalability and performance considerations.

SPARQL is the standard query language for RDF graphs, 
and despite substantial research on optimizing SPARQL query processing \cite{Pham2013}, the computation of query solutions remains a computationally intensive task for large datasets.
As a result, it becomes challenging for maintainers of SPARQL endpoints to guarantee the levels availability required in production applications such as the ones listed above.
For example, a recent survey \cite{buil2013sparql} revealed that less than half of the public endpoints reach an availability of 95\%.
The computational cost of SPARQL query answering also makes it challenging for application developers to achieve query response times of 1 or 2 seconds as users expect and as advocated in user interface guidelines \cite{nielsen1994usability}, especially when serving large numbers of users.
Application developers often spend a substantial amount of time optimizing queries to support interactive applications \cite{Loizou_Angles_Groth_2014}.

Common approaches to address the performance and availability challenges involve restricting the types of queries that can be submitted to the SPARQL endpoint.
We can think of the different approaches as defining views of an RDF graph.
A simple approach is to define views of the graph as a parameterized SPARQL queries, similar to views in a relational database.
A good example of this approach is Open PHACTS, which exposes the parameterized queries to clients using a Web API.
The approach is very conservative as it limits clients to a small number of queries.
%
Triple Pattern Fragments \cite{Verborgh2014} can be seen as providing simple views of an RDF graph, where each view is defined in terms of a triple pattern, where any of the subject, predicate or object roles can be a variable.
Queries consist of a single triple pattern, so servers can answer them with minimal computational resources.
The approach provides a client-side library that implements a subset of SPARQL in terms of Triple Pattern Fragments, thereby exposing a rich query language for clients, implemented on top of the simple, efficient views.
Each query generates multiple requests to the server, resulting in significant data transfer between servers and clients and degraded performance. 

