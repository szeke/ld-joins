\section{Introduction}
There are several examples of applications that illustrate the benefits of Semantic Web technology to build data analysis applications that integrate data from heterogeneous sources.
Notable examples include Bio2RDF \cite{callahan2013bio2rdf} with 11 billion triples, Open PHACTS \cite{Groth_Loizou_Gray_Goble_Harland_Pettifer_2014} with 3 billion triples, Event Registry \cite{Leban_Fortuna_Brank_Grobelnik_2014} with 56 million documents and DIG \cite{szekely2015building} with 74 million documents.
These systems offer rapid response times, with queries and navigation responding within one 1 or 2 seconds.
Bio2RDF and Open PHACTS integrate RDF datasets, store their data in triple stores and use SPARQL to query the data.
Event Registry and DIG integrate extractions from over 100,000 new Web documents daily.
They both use semantic technologies to integrate the data, but neither uses a triple store or SPARQL due to scalability, performance and robustness considerations.

Despite substantial research on optimizing SPARQL query processing \cite{Pham2013}, answering user queries remains a computationally intensive task for large datasets.
As a result, it is challenging for application developers to achieve query response times of 1 or 2 seconds as users expect and as advocated in user interface guidelines \cite{nielsen1994usability}, especially when serving large numbers of users.
Application developers often spend a substantial amount of time optimizing queries to support interactive applications \cite{Loizou_Angles_Groth_2014}.
In addition, it is challenging for maintainers of SPARQL endpoints to guarantee the levels availability required in production applications such as the ones listed above.
For example, a recent survey \cite{buil2013sparql} revealed that less than half of the public endpoints reach an availability of 95\%.  To address this, Triple Pattern Fragments\cite{Verborgh2014} attempts to increase availability by offloading the work from the server on to the client. 


