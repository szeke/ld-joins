\section{Introduction}
SPARQL, the standard query language for RDF graphs is also the de-facto standard for building applications that consume Linked Data. 
Despite substantial research on optimizing SPARQL query processing \cite{Pham2013}, the computation of query solutions remains a computationally intensive task for large datasets.
As a result, it becomes challenging for maintainers of SPARQL endpoints to guarantee high availability.
For example, a recent survey \cite{buil2013sparql} revealed that less than half of the public endpoints reach an availability of 95\%.
The computational cost of SPARQL query answering also makes it challenging for application developers to build high performance applications serving large numbers of users.
Application developers often spend a substantial amount of time optimizing queries to support interactive applications \cite{Loizou_Angles_Groth_2014}.

Common approaches to address the performance and availability challenges involve restricting the types of queries that can be submitted to the SPARQL endpoint.
We can think of the different approaches as defining views of an RDF graph.
A simple approach is to define views of the graph as a parameterized SPARQL queries, similar to views in a relational database.
A good example of this approach is Open PHACTS \cite{Groth_Loizou_Gray_Goble_Harland_Pettifer_2014}, which exposes the parameterized queries to clients using a Web API.
The approach is very conservative as it limits clients to a small number of queries.
%
Triple Pattern Fragments \cite{Verborgh2014} can be seen as providing simple views of an RDF graph, where each view is defined in terms of a triple pattern, where any of the subject, predicate or object roles can be a variable.
Queries consist of a single triple pattern, so servers can answer them with minimal computational resources.
The approach provides a client-side library that implements a subset of SPARQL in terms of Triple Pattern Fragments, thereby exposing a rich query language for clients, implemented on top of the simple, efficient views.
Each query generates multiple requests to the server, resulting in significant data transfer between servers and clients and degraded performance. 